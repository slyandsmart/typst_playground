\documentclass[
  type=BA, % BA, MA, PhD, or Expose
  lang=german, % english or german
  study=BauUmwelt, % BauUmwelt, Bau, Umwelt, Mech, Elek, or Doktorat
  uni=LFUI, % LFUI or UMIT
  expose=false, % true or false; defaults to false
  explanation=true, % true or false; set to false prior to printing!
  specialization=BBP, % BBP, KIB, MOS, EEG, GVW, or UVW; only applicable to MA Bau or MA Umwelt
]{ftwthesis}

% Modern LaTeX installations and editors can and should use unicode input
% encoding.
\usepackage[T1]{fontenc}
\usepackage[utf8]{inputenc}

% Using bibtex or its successor biber is the recommended way to handle your
% bibliography. You may use a bibliographystyle of your choice. This sample
% document uses natbib and a numeric citation style. Another good and modern alternative is the biblatex package.
\usepackage[numbers]{natbib}
% \usepackage[authoryear,round]{natbib} % for author-year citations

% Enter your data:
\AuthorGivenName{Bettina}
\AuthorFamilyname{Mayer}
\AcademicTitlePre{} % Mag., Dipl.-Ing., etc.
\AcademicTitlePost{BSc} % BA, MA, etc.
\setboolean{AuthorFemale}{true} % true or false

% Title, subtitle, and date (Month Year) of your thesis. Don't make titles
% excessively long!
\title{Titel der Bachelorarbeit (maximal drei Zeilen!)}
\Subtitle{Vorlage für Bachelorarbeiten an der Fakultät für Technische Wissenschaften, Universität Innsbruck}
\date{\today}


% The (principal) advisor of your thesis. This is necessary for all thesis types (BA, MA, or PhD).
\Advisor{Univ.-Prof. Dr. Anke Bockreis}
\AdvisorUniversity{Universität Innsbruck}
\AdvisorDepartment{Institut für Infrastruktur}
\AdvisorResearchUnit{Arbeitsbereich für Umwelttechnik}
\setboolean{AdvisorFemale}{true}




\begin{document}


\begin{Acknowledgment}
  \ifthenelse{\boolean{langGerman}}
  {Optional. Falls dieser Abschnitt nicht erwünscht ist, einfach löschen oder
  leer lassen.}
  {Optional. If not needed, just delete or leave empty.}
\end{Acknowledgment}

\begin{Kurzfassung}
  Kurzfassung Ihrer Arbeit in deutscher Sprache. Verpflichtend!
\end{Kurzfassung}

\begin{Abstract}
  Abstract of your thesis in English. Compulsory!
\end{Abstract}

% Insert table of contents and clear page (or double page in case of MA or PhD theses)
\tableofcontents\clearpage

% Delete next line if you do not have figures.
\listoffigures\clearpage

% Delete next line if you do not have tables.
\listoftables\clearpage

% If you wish, you may add more material such as a glossary or a list of abbreviations.


% Replace the text until >>>END with the actual contents of your thesis. The next line is only there to produce some meaningless text, a short usage info and citations for the list of references.
\FTWBlindDocument
% >>>END

% Uncomment, if you want to have an appendix.
% \appendix
% \chapter{Appendix chapter title.}

% The bibliography. Using bibtex (or the modern alternative biber in conjunction with the biblatex package) is highly recommended but not compulsory.
\bibliographystyle{plainnat}
\bibliography{ftwthesis} % replace with your own bibtex-database

\end{document}

